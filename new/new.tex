%% LyX 2.3.7 created this file.  For more info, see http://www.lyx.org/.
%% Do not edit unless you really know what you are doing.
\documentclass[11pt,oneside,english]{book}
\usepackage[T1]{fontenc}
\usepackage{geometry}
\geometry{verbose,tmargin=1in,bmargin=1in,lmargin=1in,rmargin=1in}
\setcounter{tocdepth}{1}
\usepackage{color}
\usepackage{babel}
\usepackage{amsmath}
\usepackage{amsthm}
\usepackage{graphicx}
\usepackage[unicode=true,pdfusetitle,
 bookmarks=true,bookmarksnumbered=false,bookmarksopen=false,
 breaklinks=false,pdfborder={0 0 0},pdfborderstyle={},backref=false,colorlinks=false]
 {hyperref}

\makeatletter
%%%%%%%%%%%%%%%%%%%%%%%%%%%%%% Textclass specific LaTeX commands.
\numberwithin{equation}{section}
\numberwithin{figure}{section}

%%%%%%%%%%%%%%%%%%%%%%%%%%%%%% User specified LaTeX commands.
\usepackage{titlesec}
\titleformat{\part}
  {\normalfont\sffamily\huge}
  {\partname\ \thepart}{1em}{}
\titleformat{\chapter}[display]
  {\normalfont\sffamily\huge}
  {\chaptertitlename\ \thechapter}{20pt}{\Huge}
\titleformat{\section}
  {\normalfont\sffamily\Large}
  {\thesection}{1em}{}
\titleformat{\subsection}
  {\normalfont\sffamily\large}
  {\thesection}{1em}{}
\titleformat{\subsubsection}
  {\normalfont\sffamily\normalsize}
  {\thesection}{1em}{}

\usepackage{newtxtext}
\usepackage{titletoc}

\contentsmargin[1cm]{0cm}

\titlecontents{part}[0em]{\vskip12pt\bfseries\sffamily}
{\thecontentslabel\enspace}
{\hspace{1.05em}}
{ \hfill\contentspage}[\vskip 6pt]

\titlecontents{chapter}[0em]{\vskip12pt\bfseries\sffamily}
{\thecontentslabel\enspace}
{\hspace{1.05em}}
{ \hfill\contentspage}[\vskip 6pt]

\titlecontents{section}[1em]{\sffamily}
{\thecontentslabel\enspace}
{}
{\titlerule*[1pc]{.}\quad\contentspage}[\vskip 4pt]

\titlecontents{subsection}[2.7em]{\sffamily}
{\thecontentslabel\enspace}
{}
{\titlerule*[1pc]{.}\quad\contentspage}[\vskip 3pt]

\usepackage{listings}
\usepackage{color}
\definecolor{ltgry}{rgb}{0.95,0.95,0.95}
\definecolor{purplekeywords}{rgb}{0.75,0,0.75}
\definecolor{greycomments}{rgb}{0.5,0.5,0.5}
\definecolor{redstrings}{rgb}{0.64,0.08,0.08}
\lstset{backgroundcolor=\color{ltgry}}

\lstset{columns=fixed, basicstyle=\ttfamily, basewidth=0.55em}

\usepackage{pifont}
\newcommand{\xmark}{\ding{55}}

\makeatother

\usepackage{listings}
\lstset{language=Java,
showspaces=false,
showtabs=false,
breaklines=true,
showstringspaces=false,
breakatwhitespace=true,
escapeinside={(*@}{@*)},
commentstyle={\color{greycomments}},
keywordstyle={\color{purplekeywords}\bfseries},
stringstyle={\color{redstrings}},
basicstyle={\ttfamily},
morekeywords={ var, fn, print,in }}
\renewcommand{\lstlistingname}{Listing}

\begin{document}
\title{\textsf{\emph{\includegraphics[width=3.5in]{Figures/morphologo}}}\\
\textsf{\emph{Version 0.6.1}}}

\maketitle
\newpage{}

\chapter{Variables and types}

Like many languages, \emph{Morpho} allows the programmer to define
\emph{variables} to contain pieces of information of \emph{values}.
A variable is created using the \lstinline!var! keyword, which is
followed by its name

\begin{lstlisting}
var a
\end{lstlisting}

\noindent Variable names must begin with an alphabetical character
or the underscore character \texttt{\_}, and may consist of any combination
of alphanumeric characters or underscores thereafter. Variable names
are case sensitive, so 

\noindent 
\begin{lstlisting}
var a
var A
\end{lstlisting}
each refer to distinct variables.

\noindent After creating a variable, you may immediately store information
in it by providing an \emph{initializer, }which can be any value

\begin{lstlisting}
var i = 1
var str = "Hello"
\end{lstlisting}


\section{Types}

\emph{Morpho} is a dynamically typed language: Every value has a definite
type, and Morpho is always able to tell what type it is, but variables
may generally contain values of any type and functions or methods
can accept arguments of any type. 

There are a number of basic types in Morpho:
\begin{description}
\item [{nil}] is a special value that represents the \emph{absence} of
information and is different from any other value. Unless an initializer
is provided, Morpho variables initially contain \lstinline!nil! after
declaration. 
\item [{Bool}] values contain either \lstinline!true! or \lstinline!false!. 
\item [{Int}] values contain 32 bit signed integer numbers. An integer
constant is written intuitively, e.g. \lstinline!1!, \lstinline!50!,
\lstinline!1000! and may include a negative sign \lstinline!-100!.
\item [{Float}] values contain double precision floating point numbers.
You can write numeric constants either using a decimal \lstinline!1.5!
or in scientific notation, e.g. \lstinline!1e10!, \lstinline!1.6e-19!
or \lstinline!6.625e26!.
\item [{Object}] values encompass many additional types, from Strings,
Lists, Matrices and more. 
\end{description}

\section{Strings}

Strings are sequences of unicode UTF8 encoded characters. You can
access individual characters using index notation

\begin{lstlisting}
var h = "Hello"
print h[0]
\end{lstlisting}


\section{Strings}


\end{document}
